\documentclass[a4j]{ujarticle}
\title{文章をGitで}
\author{研究室モルモット隊}

\begin{document}
\maketitle
\tableofcontents

\section{はじめに}
この部分は最初に書きました。

このあと、奥のPCと手前のPCでそれぞれ書き加えてみます。

最初にブランチを変更していなかったかも。
mainに直接書けるならちょっと嫌かも。

\section{こっちでは2章のつもりで足す}
3章になるのか、重なって書かれるのか。
手前だとまだ同期できない。チェックアウトして上に行ってからマージ?

メッセージは送れたけどコミットができない?

イグノアの設定がいると分かった。
でも変更にまだ出てくるか。

\section{2限の合間に追加}
ノートPCでも接続できた。

\section{ignore}
tex以外のファイルを無視してくれないなあ

\section{研究室のPCに座って}
20240612 1908

\TeX の文書だけなら結構順調?
自分一人なら調整しやすくなるか

\end{document}